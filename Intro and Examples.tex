\title{Groups 4 Subgroups}
\date{}
\author{}
\documentclass{article}
\usepackage{amsmath,amsfonts,amssymb}
\usepackage[usenames,dvipsnames]{color}
\usepackage{amsthm, thmtools}
\usepackage{fullpage, changepage, tabto}
\usepackage{enumerate, enumitem}
\usepackage{mathtools}
\usepackage[colorlinks=true, linkcolor=BlueGreen, citecolor=BlueGreen, urlcolor=BlueGreen]{hyperref}
\usepackage{asymptote}
\usepackage{wrapfig}
\usepackage{parskip}

% set tab distances
\NumTabs{16}

%%% theorem environments for styling
% notation
\declaretheoremstyle[
    notefont={\normalfont},
    spaceabove={0.5em},
    spacebelow={0.5em},
    notebraces={}{},
    headformat={\NOTE},
    headpunct={},
    postheadhook={\hspace{0em}\tabto{4em}},
    ]{notation}
\declaretheorem[style=notation]{notation}

% definition
\declaretheoremstyle[
    headfont={\itshape},
    notefont={\normalfont\bfseries},
    spaceabove={1em},
    notebraces={}{},
    headformat={def \NOTE \newline},
    headpunct={ },
    postheadhook={\hspace{0em}\vspace{0.25em}\tab},
    ]{definition}
\declaretheorem[style=definition,name=Definition,
refname={definition,definitions}]{definition}

% note
\declaretheoremstyle[
    headfont={\bfseries},
    shaded={margin={1.5em}, textwidth={20em}},
    headformat={note: \newline},
    headpunct={ },
    ]{note}
\declaretheorem[style=note]{note}


% prettier empty set
\newcommand{\nullset}{\varnothing}

% logic spacing
\newcommand{\spaced}[1]{\, #1 \,}
\newcommand{\scomp}{\spaced{\backslash}}
\newcommand{\sland}{\spaced{\land}}
\newcommand{\slor}{\spaced{\lor}}
\newcommand{\simplies}{\spaced{\Rightarrow}}
\newcommand{\slexists}{\, \exists}

% set macros
\newcommand{\buildset}[2]{\{\spaced{#1} \mid \spaced{#2} \}}
\newcommand{\finiteset}[3]{\{ \spaced{#1,} \spaced{#2,} \spaced{#3,} \spaced{\dots} \}}

\begin{document}
    \maketitle
    \tableofcontents
    \newpage
    \section{intro}
    $$\text{Consider }U = \buildset{z \in \mathbb{C}}{\lvert z \rvert = 1} \subseteq \mathbb{C}.$$
    \begin{wrapfigure}[2]{l}{0.35\textwidth}
    \begin{asy}
    settings.outformat="pdf";
    unitsize(1cm);
    draw((0,0) -- (2,0), arrow=Arrow(TeXHead));
    draw((0,0) -- (-2,0), arrow=Arrow(TeXHead));
    draw((0,0) -- (0,-2), arrow=Arrow(TeXHead));
    draw((0,0) -- (0,2), arrow=Arrow(TeXHead));
    draw(unitcircle);
    \end{asy}
    \end{wrapfigure}
    \vspace{24pt}
    \begin{note}
    each elem of $U$ is defined by $\theta \in [0, 2\pi) = \mathbb{R}_{2\pi}$
    \end{note}
    \vspace{48pt}
    \begin{center}
    Every angle $\theta \in \mathbb{R}_{2\pi}$ given by $f: U \rightarrow R_{2\pi}, \, f(z) = \theta$ for $z = e^{i\theta}$\\
    \vspace{8pt}
    +\\
    \vspace{8pt}
    $f^{-1}: \mathbb{R}_{2\pi} \rightarrow U, \, f^{-1}(\theta) = e^{i\theta}$\\
    \vspace{24pt}
    Consider $U$ with multiplication:\\
    \medskip
    Let $z_1 = e^{i\theta_1},\, z_2 = e^{i\theta_2} \in U$\\
    \vspace{8pt}
    then $z_1 \cdot z_2 = e^{i\theta_1}\cdot e^{i\theta_2} = e^{i\left(\theta_1 + \theta_2\right)}$ where $\theta_1, \, \theta_2 \in \mathbb{R}_{2\pi},\, \theta_1 + \theta_2 \in \mathbb{R}_{2\pi}$\\
    \vspace{8pt}
    Set $U$ is \underline{closed} under multiplication. 
    \end{center}

\section{definition}
    \begin{definition}[algebraic structure]\label{algebraic structure}
    \hspace{0em}\vspace{-1em}\\
    \tab\tab Let $S$ be a set.\\
    \tab\tab Let $\star$ be a binary operation on $S$\\
    \tab\tab\tab $\star: S \times S \rightarrow S$\\
    \tab\tab $\left(S,\, \star \right)$ is an \textbf{algebraic structure}
    \end{definition}
    \vspace{12pt}
    So $\left(U,\, \cdot\right)$ and $\left(\mathbb{R}_{2\pi},\, +_{2\pi}\right)$ are algebraic structures such that there is a ``1-1" relation between $U$ and $\mathbb{R}_{2\pi}$ and operations are preserved.
    \vspace{8pt}
    
    $$\left(U,\, \cdot\right) \simeq \left(\mathbb{R}_{2\pi},\, +_{2\pi}\right)$$
    \begin{center}or\end{center}
    $$U \simeq \mathbb{R}_{2\pi}$$

\section{isomorphisms}
To show two algebraic structures are isomorphic, find a bijective function that preserves the operations.\\
Isomorphic structures share the same algebraic properties (associativity, commutative, distributive...).\\
To show two algebraic structures are \underline{not} isomorphic, find a property that they don't share (the easiest to check is cardinality).

\subsection{Examples}
\begin{center}
$$\left(\mathbb{N},\, +\right) \not\simeq \left(\mathbb{Z},\, +\right)$$
because $\mathbb{N}$ has no additive identity
$$\left(\mathbb{Z}^{*},\, \cdot\right) \not\simeq \left(\mathbb{Q}^{*},\, \cdot\right)$$
because not all elements of $\mathbb{Z}^*$ have inverses $\left(\frac{n}{m}\right)^{-1} = \frac{m}{n}$
$$\left(\mathbb{R},\, +\right) \not\simeq \left(\mathbb{R},\, \cdot\right)$$
because $x \star x + 1 = 0$ is solveable in $\left(\mathbb{R},\, +\right): 2x + 1 = 0$ but unsolveable in $\left(\mathbb{R},\, \cdot\right): x^2 + 1 = 0$
\end{center}        
\end{document}
