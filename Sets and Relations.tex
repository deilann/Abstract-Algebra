\title{Sets 4 Relations}
\date{}
\author{}
\documentclass{article}
\usepackage{amsmath,amsfonts,amssymb}
\usepackage[usenames,dvipsnames]{color}
\usepackage{amsthm, thmtools}
\usepackage{fullpage, changepage, tabto}
\usepackage{enumerate, enumitem}
\usepackage{mathtools}
\usepackage[colorlinks=true, linkcolor=BlueGreen, citecolor=BlueGreen, urlcolor=BlueGreen]{hyperref}

% set tab distances
\NumTabs{16}

%%% theorem environments for styling
% notation
\declaretheoremstyle[
    notefont={\normalfont},
    spaceabove={0.5em},
    spacebelow={0.5em},
    notebraces={}{},
    headformat={\NOTE},
    headpunct={},
    postheadhook={\hspace{0em}\tabto{4em}},
    ]{notation}
\declaretheorem[style=notation]{notation}

% definition
\declaretheoremstyle[
    headfont={\itshape},
    notefont={\normalfont\bfseries},
    spaceabove={1em},
    notebraces={}{},
    headformat={def \NOTE \newline},
    headpunct={ },
    postheadhook={\hspace{0em}\vspace{0.25em}\tab},
    ]{definition}
\declaretheorem[style=definition,name=Definition,
refname={definition,definitions}]{definition}

% note
\declaretheoremstyle[
    headfont={\bfseries},
    shaded={margin={1.5em}, textwidth={20em}},
    headformat={note: \newline},
    headpunct={ },
    ]{note}
\declaretheorem[style=note]{note}


% prettier empty set
\newcommand{\nullset}{\varnothing}

% logic spacing
\newcommand{\spaced}[1]{\, #1 \,}
\newcommand{\scomp}{\spaced{\backslash}}
\newcommand{\sland}{\spaced{\land}}
\newcommand{\slor}{\spaced{\lor}}
\newcommand{\simplies}{\spaced{\Rightarrow}}
\newcommand{\slexists}{\, \exists}

% set macros
\newcommand{\buildset}[2]{\{\spaced{#1} \mid \spaced{#2} \}}
\newcommand{\finiteset}[3]{\{ \spaced{#1,} \spaced{#2,} \spaced{#3,} \spaced{\dots} \}}

\begin{document}
    \maketitle
    \tableofcontents
    \newpage
    \section{notation}
        \vspace{1em}
        % --- changes in some sections to update note shorthand to clear text
        \begin{notation}[$\mathbb{R}$]
            the set of real numbers
        \end{notation}
        \begin{notation}[$\mathbb{C}$]
            the set of complex numbers
        \end{notation}
        \begin{notation}[$\mathbb{Q}$]
            the set of rational numbers \tab\tab $\buildset{\frac{n}{m}}{n, m \in \mathbb{Z} \sland m \not= 0 \sland \gcd(n, m) = 1}$
        \end{notation}
        \begin{notation}[$\mathbb{Z}$]
            the set of integers
        \end{notation}
        \begin{notation}[$\mathbb{N}$]
            the set of natural numbers \tab\tab $\finiteset{1}{2}{3}$
        \end{notation}
        \begin{notation}[$A^*$]
            the non-zero elements of $A$ \tab\tab $\buildset{a \in A}{a \not= 0}$
        \end{notation}
        \begin{notation}[$A^+$]
            the positive elements of $A$ \tab\tab
            $\buildset{a \in A}{a > 0}$
        \end{notation}
        \begin{notation}[$A^-$]
            the negative elements of $A$ \tab\tab
            $\buildset{a \in A}{a < 0}$
        \end{notation}
        \begin{notation}[$\mathbb{R} \scomp \mathbb{Q}$]
            the set of irrational numbers \tab\tab $\buildset{a \in \mathbb{R}}{a \notin \mathbb{Q}}$
            \tab note: $(\mathbb{R} \scomp \mathbb{Q}) \cap \mathbb{Q} = \nullset$
        \end{notation}
        \begin{notation}[$\exists x$]
            there exists at least one $x$
        \end{notation}
        \begin{notation}[$\exists 'x$]
            there exists only one $x$
        \end{notation}
    
    \newpage
    \section{set definitions}
        \vspace{1em}
        
        \color{blue}
        \begin{definition}[subset]\label{subset}
            $A$ is a \textbf{subset} of $B$ $A \subseteq B$ iff $\forall a \in A, \, a \in B$
        \end{definition}
        
        \color{ForestGreen}
        \begin{definition}[proper subset]\label{proper subset}
            % ---changed from the notes which said
            % $A$ is a \textbf{proper subset} of $B$ iff $\forall a \in A,
            % a \in B \land \exists b \in B, b \in A$ (denoted $A \subset B$)
            $A$ is a \textbf{proper subset} of $B$ iff $\forall a \in A, a \in B \sland \slexists b \in B, b \not\in A$ (denoted $A \subset B$)
        \end{definition}
        
        \color{Red}
        \begin{definition}[relation]\label{relation}
            a \textbf{relation} $\mathcal{R}$ on a set $A$ is a subset of $A \times A$ \tab\tab $\mathcal{R} \subseteq A \times A$
        \end{definition}
        
        \color{Purple}
        \begin{definition}[equivalence relation]\label{equivalence relation}
            $\mathcal{R}$ is an \textbf{equivalence relation} iff
            \begin{enumerate}[label={\arabic*)}, partopsep=0.5em, itemsep=0.25em, left=6em]
            
                \item it is \textbf{reflexive}
                \begin{definition}[reflexive]\label{reflexive}
                    $\forall a \in A, \; a\mathcal{R}a \quad ((a, a) \in \mathcal{R})$\end{definition}
                % --- changed from notes which had for all a, a in A
                
                \item it is \textbf{symmetric}
                \begin{definition}[symmetric]\label{symmetric}
                    $\forall \, a, \, b \in A, \; a\mathcal{R}b \simplies b\mathcal{R}a$
                \end{definition}
                
                \item it is \textbf{transitive}
                \begin{definition}[transitive]\label{transitive}
                    $\forall \, a, \, b, \, c \in A, \; a\mathcal{R}b \sland b\mathcal{R}c \simplies a\mathcal{R}c$
                \end{definition}
                
            \end{enumerate}
        \end{definition}
        
    \color{Black}    
    \newpage
    \section{set operations}
    
        \begin{notation}[$A \cup B$]
            union \tab\tab\tab\tab\tab\tab $\buildset{x}{x \in A \slor x \in B}$
        \end{notation}
        
        \begin{notation}[$A \cap B$]
            intersection \tab\tab\tab\tab\tab $\buildset{x}{x \in A \sland x \in B}$
        \end{notation}
        
        \begin{notation}[$A \scomp B$]
            complement of $B$ in $A$ \tab\tab\tab $\buildset{x}{x \in A \sland x \not\in B}$
        \end{notation}
        
        \begin{notation}[$A \times B$]
            cartesian product of $A$ and $B$ \tab $\buildset{(x,\, y)}{x \in A \sland x \in B}$
        \end{notation}
        
        \vspace{1em}
        \begin{note}
            \hspace{1em}\vspace{1em}
                $$\begin{rcases*}
                    A \cup B = B \cup A \quad \\
                    A \cap B = B \cap A \quad \\
                \end{rcases*} \quad \text{commutative}$$
                $$\quad\quad A \times B \not= B \times A \quad \big\rbrace \quad \text {not commutative}$$
        \end{note}
        \vspace{1em}

    \newpage
    \section{properties of relations}
        % --- reordered these for ease of reading
        
        \color{Red}
        \begin{definition}[injective]\label{injective}
            a function $f: \, A \rightarrow B$ is 
            \textbf{injective} (one-to-one) iff $\forall \, a_1, a_2 \in A, \, f(a_1) = f(a_2) \simplies a_1 = a_2$
        \end{definition}
        
        \begin{definition}[surjective]\label{surjective}
            $f: \, A \rightarrow B$ is 
            \textbf{surjective} (onto) iff $\forall b \in B \, \slexists a \in A \text{ where } f(a) = b$
        \end{definition}
        
        \begin{definition}[bijective]\label{bijective}
            a function $f: \, A \rightarrow B$ is \textbf{bijective} iff it is 
            \nameref{injective} (one-to-one) and \nameref{surjective} (onto $B$).
            \color{Black}
            
            \begin{note}
                \tab a bijective function has an inverse
            \end{note}
        \end{definition}
        
    \newpage
    \color{Black}
    \section{cardinality}
    
        \begin{definition}[equinumerous]\label{equinumerous}
            \hspace{0em}\vspace{-1em}\\
            \tab\tab having the same cardinality (size)\newline
            \vspace{1em}
            \tab let $S$ be a collection of sets and $\mathcal{R}$ a \nameref{relation} on $S$.\newline
            \vspace{1em}
            \tab let $A,\, B$ be sets and $A,\, B \in S$. $A\mathcal{R}B$ iff $\exists f$ where $f: A \rightarrow B$ and $f$ is \nameref{bijective}. if $A\mathcal{R}B$, $A$ and $B$ are \newline
            \tab\tab equinumerous because they have a one-to-one correspondence\newline
            \vspace{1em}
            \tab consider:
            
            \begin{proof}[\unskip\nopunct]
                \begin{enumerate}[label={\arabic*)}, partopsep=1em, itemsep=0.25em, left=6em]
                
                    \item $\forall A \in S, \, A\mathcal{R}A$ because $\exists f = \text{id}_{A}: A \rightarrow A$ as $f$, bijective, and so $\mathcal{R}$ is \nameref{reflexive}
                    
                    \item $\forall A, \, B \in S$, $A\mathcal{R}B \simplies B\mathcal{R}A$ because $\exists f: A \rightarrow B$ and as $f$ is bijective $\exists g = f^{-1}: B \rightarrow A$, so $\mathcal{R}$ is \nameref{symmetric}
                    \item $\forall A, \, B, \, C \in S, \, A\mathcal{R}B \sland B\mathcal{R}C \simplies A\mathcal{R}C$ as $\exists f: A \rightarrow B,\slexists g: B \rightarrow C$ and $f,\, g$ are bijective $\Rightarrow \slexists h: A \rightarrow C$ so $\mathcal{R}$ is \nameref{transitive} 

                \end{enumerate}
                
                \vspace{1em}
                \tab thus $\mathcal{R}$ is an \nameref{equivalence relation} on $S$ and partitions $S$
                \end{proof}
                
                \begin{note}
                    \hspace{0em}\vspace{0.5em}
                    \tab $E_A = \buildset{\mathcal{R} \in S}{A\mathcal{R}B}$ contains all the \newline
                    \tab\tab sets of size $|A|$
                \end{note}
                
        \end{definition}
        
        \begin{definition}[denumerable]\label{denumerable}
            a set $A$ is \textbf{denumerable} iff $A$ is \nameref{equinumerous} to $\mathbb{N}$
        \end{definition}
        
        \begin{definition}[countable]\label{countable}
            a set $A$ is \textbf{countable} iff $A$ is finite or \nameref{denumerable}
        \end{definition}
        
\end{document}
